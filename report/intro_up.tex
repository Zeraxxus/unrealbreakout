\section{Unified Process}
Unified Process forkortes også UP og er en systemudviklingsmetode. Det blev udviklet i slutningen af 90'erne af Ivar Jacobsen, Grady Booch og James Rumbaugh. I Unified Process gøres der brug af iterationer som programmet skal udvikles i, et større projekts iterationer varer typsik 2 til 6 uger. Når man udvikler et stykke software med Unified Process skal man igennem fire faser. Inception fasen hvor man forbereder hvilke ting der skal laves og danner et overblik over de forskellige opgaver der skal løses. De næste fase kaldes Elaboration, i denne fase udvikles de centrale dele af programmet samt de dele der er sværest at lave, da de skal udvikles tidligt i processen så de ikke bliver udviklet under tidspres. Elaboration fasen er længere end Inception fasen og er også den første fase hvor programmet testes. Efter Elaboration kommer Construction fasen, her udvikles de elementer i systemet som skal til for at programmet bliver færdigt, samt at programmet testes løbende igennem fasen.