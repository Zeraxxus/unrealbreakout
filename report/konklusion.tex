\rhschapter{Konklusion}

Det lykkedes os at lave en prototype af et breakout type spil i Unreal Engine 4.

Under analysefasen har vi brugt use cases, analyse-klassediagram, system definition(FACTOR), hændelsestabel, funktionsliste. Disse har været meget hjælpsomme til hurtigt at skyde projektet i den rigtige retning. Dette sørger for at alle i projektet har samme idé om hvad der udvikles.
Til design af spillet har vi lavet en gameplay-beskrivelse og hentet grafik fra \textit{opengameart.org}.

Unified Process har givet os rigtig gode værktøjer til at planlægge projektets forløb med, så man ved hvor meget tid og arbejdskraft man har at gøre med. Hvis noget i tidsplanen så skrider, har vi hurtigt kunne reagere på det.

Til at lave selve spillet har vi brugt Unreal Engine 4 med blueprint-, sprite-, og level-editor. Sprite-editoren har gjort det let for os automatisk at splitte spritesheet billedfiler op i mindre elementer. Level-editoren har gjort det let at lave hovedmenuen og banen ved visuelt at placere objekter. Blueprint-editoren er en visuel måde at repræsentere kode(scripts), ved at oprette noder med specifik funktionalitet og forbinde dem sammen.