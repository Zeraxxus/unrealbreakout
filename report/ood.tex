\rhschapter{Object Orienteret Design}

\section{Gameplay}
Unreal Breakout er et spil med et bat\footnote{Paddle på engelsk, bliver brugt i vores paddle klasse.}, en bold\footnote{Ball på engelsk, bliver brugt i vores ball klasse.}, en masse brikker\footnote{Bricks på engelsk, bliver brugt i vores bricks klasse.}, tre lukkede sidder, og en åben side. Formålet er at skyde bolden op og ødelægge alle brikkerne og sørge for at bolden ikke ryger ud af banen i bunden hvor spillerens bat og den åbne side er. Når man rammer en brik, går den i stykker eller skifter farve, og så får man point tilføjet sin score. Når spillet starter har man 3 forsøg(bolde) og bolden er låst fast på éns bat i midten. Når man trykker på en dertil bestemt knap skyder man bolden afsted, og med nogle andre knapper bevæger man battet fra side til side for at undgå at bolden ryger ud af spilområdet. Hvis bolden ryger ud af spilområdet, mister man et forsøg(bold) og en ny bold bliver låst fast til batet og klar til at blive skudt afsted på ny. Når man har opbrugt alle forsøg og den sidste bold ryger ud af spilområdet, slutter spillet og man kan se sin score. Får man til gengæld alle brikker fjernet vinder man banen og den samme bane kommer igen uden at ændre på antal forsøg man har tilbage eller nulstille éns score, og bolden bliver igen fastlåst til batet.

\section{Grafik}

Spillets grafik er fundet på \textit{opengameart.org}, som er en hjemmeside hvor folk lægger grafik op til fri afbenyttelse. Dette betyder at det er uden licens men man kan dog donere penge til de brugere der har lagt grafikken tilgængelig på siden. \newline
Breakout er et gammelt spil udviklet af Atari og derfor er der fundet grafik som er tro mod det klassiske spils grafik. Grafikken ligger nogle spritesheets som er kollager af billeder som bliver brugt til at skabe de forskellige elementer i spillet. Når spritesheetet er delt op vil de forskellige brikker blive bygget op til et level. Eftersom spilleren rammer brikkerne med bolden vil de skifte farve, farverne indikere hvor mange gange en brik skal rammes for at blive ødelagt. De grafiske elementer er ikke noget der vil blive sat stor fokus på i spillet, da det er spillets gameplay der skal være grundpillen i det. 
\section{Blueprints}
Blueprints er en speciel type asset som giver en node baseret interface til at lave nye klasser og udvide klasser, så man kan scripte objekter i levels og widgets. Blueprints er et værktøj der giver designers og gameplay programmører mulighed for hurtigt at kunne iterere deres idéer uden at skulle skrive kode.

Det er stadig mulig selv at oprette klasser med C++ kode uden at bruge blueprints. Blueprints kan bruges til at udvide klasser som et skrevet i C++, gemme og modificeret properties, og håndtere objekt instancer i klasser.

Ligesom i C++ og C\# har blueprints member variables/fields, member functions, og en constructor.

\begin{list}{}{}
\item[Der er 3 typer Blueprints:]
\item[Blueprint Class]
\item[Data-Only Blueprint]
\item[Level Blueprint]
\end{list}

Blueprint Class er den blueprint man bruger når man vil have et object til at gøre noget inde i en level. Man laver noget funktionalitet i den og tilføjer den til et objekt i en level. Data-Only Blueprints indeholder kun nodes, variabler og componenter som den har nedarvet og der kan ikke tilføjes nyt. Det den bruges til er at ændre på objekter i en level og lave variationer som alle har samme interface, men forskellig adfærd. Level Blueprint er en blueprint som altid findes når man opretter en ny level. Det er et overordnet blueprint som eksisterer sammen med de Blueprints som sidder på objekter i banen. Man kan bruge Level Blueprint til at oprette events som andre objekters blueprints kan agere på. Man kan også oprette instancer af andre assets gennem Level Blueprint, som f.eks. et UI der skal tegnes i Camera Actor'en i banen. Se \cite{blueprint} for mere information.
