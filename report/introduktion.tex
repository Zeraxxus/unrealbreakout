\rhschapter{Introduktion}

% up
\section{Unified Process}
Unified Process forkortes også UP og er en systemudviklingsmetode. Det blev udviklet i slutningen af 90'erne af Ivar Jacobsen, Grady Booch og James Rumbaugh. I Unified Process gøres der brug af iterationer som programmet skal udvikles i, et større projekts iterationer varer typsik 2 til 6 uger. Når man udvikler et stykke software med Unified Process skal man igennem fire faser. Inception fasen hvor man forbereder hvilke ting der skal laves og danner et overblik over de forskellige opgaver der skal løses. De næste fase kaldes Elaboration, i denne fase udvikles de centrale dele af programmet samt de dele der er sværest at lave, da de skal udvikles tidligt i processen så de ikke bliver udviklet under tidspres. Elaboration fasen er længere end Inception fasen og er også den første fase hvor programmet testes. Efter Elaboration kommer Construction fasen, her udvikles de elementer i systemet som skal til for at programmet bliver færdigt, samt at programmet testes løbende igennem fasen.

% process
\section{Process}
\textit{Af Nichlas Bruun}\newline
I udviklingsforløbet er der planlagt at benytte Unified Process, og eksamens projektet kommer til at illustrerer tre små UP-iterationer. 
I disse iterationer vil der være fokus på planlægning og kode. Planlægningen vil bestå både af UP-, objektorienteret analyse- og objektorienterede design-værktøjer. Nogle af de værktøjer der vil være taget i brug, er f.eks. et Gantt-chart til tidsplanlægning og et analyse-diagram til overblik af objekter i programmet. Disse værktøjer og deres brug, vil blive beskrevet i systemudviklingsrapporten. Efter planlægningen og rapportskrivningen vil udviklingsprocsessen træde i kraft, hvor selve spillet vil blive udviklet. Efter udviklingsprocessen vil der være en kort testfase, hvor diverse fejl og mangler vil blive udredt. Til sidst vil udviklingsprocesen samt refleksioner og konklusioner blive skrevet i systemudviklingsrapporten. En kort opsummering i opremsning: Planlægning, rapportskrivning, udvikling, test og til sidst rapportskrivning.


% versionsstyring
\section{Versionsstyring}
Versionsstyring er et værktøj, der gør det nemt for flere personer at arbejde på de samme filer, og gør at der er backups man kan vende tilbage til. Ved versionsstyring er der altid mulighed for at tage en ældre version af filerne, hvis der skulle være
noget som er gået galt i en nyere version. Når der er to eller flere som sidder og arbejder på samme program, gør versionsstyring det muligt at sammenflette text baserede filer automatisk, men også manuelt skulle der være konfikter på linier.

Vi har anvendt versionstyring til udarbejdelsen af rapporten, men ikke spillet. Rapporten består af mange rå tekst filer, og disse er optimale at anvende versionstyring på. Spillet består til gengæld af blueprints og levels til Unreal Engine 4, og ved disse har vi ikke selv kontrol over tekst linier som ved ren kode. Derfor kan vi ikke bruge vores egne versionsprogrammer til at sammenflette koden mellem 2 udgaver af den samme level eller blueprint. Unreal engine har sit eget versionstyring integreret, så hvis man skal bruge det skal man tilmelde sig en tjeneste der passer til det. ***reference/påstand***.
Vi har valgt at bruge Github som depot for vores rapport tekst, da det er gratis at anvende ved offentlige projekter. Skal koden være privat skal man betale for en opgraderet bruger. Github har også selv deres egen git klient som man kan bruge, men man kan også bruge andre programmer der understøtter git protokollen.***reference/påstand***.

% ooad
\section{Objekt Orienteret Analyse og Design}
Efter problemformuleringen, afgrænsningen  og spil typen; Breakout var bestemt, satte udviklingsteamet sig sammen og lavede et analysediagram. Analysediagrammet er et godt værktøj til, at få alle udviklere på samme spor og tankegang omkring projektet. Analysediagrammet lister produktets objekter, og deres relationer til hinanden. På denne måde fik udviklingsholdet snakket sammen om forholdet imellem de forskellige objekter, hvilket vil lede til en mere hensigtsmæssig udvikling, da hele udviklingsholdet får samme forståelse for hvad hvert objekt kommer til at indebære. Efter dette blev use cases udviklet, som beskriver hvilke funktioner brugeren af produktet har adgang til. Use case-værktøjet udpensler derfor hvilke funktioner produktet skal indeholde set fra brugerens synspunkt. Dette giver udviklingsholdet en sans for hvilke funktionaliteter har en sammenhæng mellem objekter. For at gøre dette helt klart, udvikles en hændelses- og funktionstabel. Disse værktøjer udmærker sig ved, at klarificere hvilke objekter i systemet der har fælles funktionalitet. Udover dette bruges værktøjerne også til at specificere funktionstyper, og kompleksiteten af disse. Til sidst i denne fase laves en systemdefinition, med værktøjet "FACTOR" til hjælp. Systemdefinitionen er en kort tekst der beskriver systemets facetter, dette kan bl.a. være systemets ansvar over for brugeren eller teknologien benyttet til at udvikle systemet. Efter brugen af alle disse værktøjer, er systemet til udvikling klarlagt og udviklingen kan begyndes.
