\rhschapter{Object Orienteret Analyse}

\section{Systemdefinition}

\section{Funktionstabel}
Funktions tabel.
\begin{table}[]
\centering
\caption{My caption}
\label{my-label}
\begin{tabular}{lll}
\multicolumn{1}{c}{Navn}          & \multicolumn{1}{c}{Kompleksitet} & \multicolumn{1}{c}{Type} \\ \hline
\multicolumn{1}{l|}{Start Game}   & \multicolumn{1}{l|}{simpel}      & update                   \\
\multicolumn{1}{l|}{Exit Menu}    & \multicolumn{1}{l|}{simpel}      & update                   \\
\multicolumn{1}{l|}{Exit Game}    & \multicolumn{1}{l|}{simpel}      & update                   \\
\multicolumn{1}{l|}{Release Ball} & \multicolumn{1}{l|}{medium}      & update / compute         \\
\multicolumn{1}{l|}{Bounce Ball}  & \multicolumn{1}{l|}{medium}      & update / compute         \\
\multicolumn{1}{l|}{Move Paddle}  & \multicolumn{1}{l|}{simpel}      & update / compute         \\
\multicolumn{1}{l|}{Break Brick}  & \multicolumn{1}{l|}{simpel}      & update                   \\
\multicolumn{1}{l|}{Change Score} & \multicolumn{1}{l|}{simpel}      & update / compute         \\
\multicolumn{1}{l|}{Show Score}   & \multicolumn{1}{l|}{simpel}      & read                    
\end{tabular}
\end{table}

\section{Klasse Diagram}

\section{Eventtabel}
Event tabel.
\begin{table}[]
\centering
\caption{My caption}
\label{my-label}
\begin{tabular}{l|ccccccc}
\multicolumn{1}{c}{Events/Klasser} & Paddle & Ball & Brick & Menu & Gameworld & Camera & UI \\ \hline
Player clicks on menu obj &                            &                          &                           & X                        &                               &                            &                        \\
Player moves paddle          & X                          &                          &                           &                          & X                             &                            &                        \\
Player releases ball         & X                          & X                        &                           &                          & X                             &                            &                        \\
Ball collides with object    & X                          & X                        & X                         &                          & X                             &                            &                        \\
Score updates                &                            & X                        & X                         &                          & X                             &                            & X                     
\end{tabular}
\end{table}

\section{Use Cases}

\textbf{Start Spillet:}\newline

\textbf{Scope:}\newline
i menuen.\newline

\textbf{Description:} \newline
Spilleren trykker på \textit{play}-knappen, med musen, i menuen for at komme til \textit{gameworlden}. \newline

\textbf{Preconditions:}\newline
Spilleren skal befinde sig i menuen for at use casen kan startes.\newline

\textbf{Success Guarantee:}\newline
Når use casen er opfyldt vil spillet gå fra menuen til \textit{gameworlden}.\newline

\textbf{Main Success Scenario:}\newline
Spilleren befinder sig i menuen og klikker med musen på \textit{play}-knappen og \textit{gameworlden} vises frem på skærmen.\newline

\textbf{Extensions:}\newline
Befinder spilleren sig allerede i \textit{gameworlden} vil det ikke være muligt at trykke på play knappen da den ikke er vises.\newline \newline



\textbf{Luk Spillet.}\newline

\textbf{Scope:}\newline
i menuen. \newline

\textbf{Description:} \newline
Spilleren kan lukke spillet ved at trykke på \textit{exit}-knappen med musen.\newline

\textbf{Preconditions:}\newline
Spilleren skal befinde sig i menuen for at use casen kan startes.\newline

\textbf{Success Guarantee:}\newline
Når use casen er opfyldt vil spillet blive lukket. \newline

\textbf{Main Success Scenario:}\newline
Spilleren befinder sig i menuen og klikker på \textit{exit}-knappen, med musen, hvorefter spillet vil lukke.\newline

\textbf{Extensions:}\newline
Befinder spilleren sig i \textit{gameworlden} vil \textit{exit}-knappen ikke være vist og spilleren kan derfor ikke interagere med den. \newline \newline


\textbf{Bevægelse af paddle.}\newline

\textbf{Scope:}\newline
I \textit{Gameworld}.\newline

\textbf{Description:} \newline
Spilleren kan ved hjælp af højre og venstre piletast bevæge \textit{paddlen} til henholdsvis højre og venstre.\newline

\textbf{Preconditions:}\newline
Spilleren har startet spillet igennem menuen.\newline

\textbf{Success Guarantee:}\newline
Bliver use casen opfyldt korrekt vil \textit{paddlen} blive bevæget til siderne i takt med at spilleren trykker piletasterne ned.\newline

\textbf{Main Success Scenario:}\newline
 a. Spilleren trykker højre piletast ned og \textit{paddlen} bevæger sig til højre.\newline
 b. Spilleren trykker venstre piletast ned og \textit{paddlen} bevæger sig til venstre.\newline \newline


\textbf{Skyd.}\newline

\textbf{Scope:}\newline
I \textit{Gameworld}.\newline

\textbf{Description:} \newline
Bolden skydes væk fra \textit{paddlen} ved brug af \textit{space}-knappen, dette kan kun gøre når spillet startes eller efter bolden har været uden for spilbanen. \newline

\textbf{Preconditions:}\newline
Spilleren har startet spillet igennem menuen og har ikke skudt endnu, eller bolden har lige været udenfor banen.\newline

\textbf{Success Guarantee:}\newline
Bliver use casen opfyldt korrekt vil bolden blive skudt væk fra \textit{paddlen}.\newline

\textbf{Main Success Scenario:}\newline
Spilleren har startet spillet og bolden er stadig placeret på \textit{paddlen}, eller bolden har lige været uden for spilbanen. spilleren trykker på \textit{space}-knappen og bolden bliver skudt væk fra \textit{paddlen}.\newline

\textbf{Extensions:}\newline
Bolden er blevet skudt væk fra \textit{paddlen} og har ikke været udenfor spilbanen og kan derfor ikke blive skudt fra \textit{paddlen} igen da den ikke er placeret der.\newline \newline


\textbf{Gå tilbage til menuen.}\newline

\textbf{Scope:}\newline
i \textit{Gameworld}.\newline

\textbf{Description:} \newline
Spilleren går tilbage til menuen fra \textit{gameworlden} ved at trykke på \textit{escape}-knappen.\newline

\textbf{Preconditions:}\newline
Spilleren har startet spillet og befinder sig i \textit{gameworlden}.\newline

\textbf{Success Guarantee:}\newline
Bliver use casen opfyldt korrekt vil spilleren blive vist menuen igen og \textit{gameworlden} vil lukke.\newline

\textbf{Main Success Scenario:}\newline
Spilleren har startet spillet og trykker på \textit{escape}-knappen, herefter vil menuen blive vist og \textit{gameworlden} vil stoppe. \newline

\textbf{Extensions:}\newline
Befinder spilleren sig allerede i menuen vil \textit{escape}-knappen ikke have nogen effekt.\newline