\section{Dijkstra}
Professor Edsger Wybe Dijkstra har bidraget med mange nyttige ting til computer videnskab. En af disse kaldes Dijkstra's Algorithm og er en pathfinding algoritme der gør brug af shortest path trees. Som man kan se på figur \ref{dia:shortest_path_trees}, har en edge imellem noder en kost associeret med at vælge den. Derved kan man lave et træ af veje fra node til node, og beregne hvilken vej der er den billigste(korteste) at bruge. Se \textcite[233]{buckland}, \textcite{amitastar}, \textcite[204]{MillingtonFunge}.

\begin{figure}
	\begin{center}
		\includegraphics[width=0.49\linewidth]{pictures/pathfinding/shortest_path_tree}
		\caption{Shortest Path Trees: Deler sig hver gang der er flere retninger. Hvis de rammer den samme node, er det den korteste af dem der fortsætter.}
		\label{dia:shortest_path_trees}
	\end{center}
\end{figure}

Hvis man kigger på implementeringen af Dijkstra's Algorithm i figur \ref{dia:dijkstra1}, kan man se at den laver en cirkelformet søgning ud fra startpunktet til den møder målet. Læg også mærke til at ingen af vejene krydser hinanden, for når algoritmen rammer en node, der allerede er i træet, bliver begge to evalueret, og kun den korteste af de to vil der blive brugt til videre beregninger. Læg også mærke til at det er dyrere at bevæge sig diagonalt end horisontalt og vertikalt med algoritmen.

\begin{figure}
	\begin{center}
		\includegraphics[width=0.49\linewidth]{pictures/pathfinding/dijkstra1}
		\includegraphics[width=0.49\linewidth]{pictures/pathfinding/dijkstra2}
		\caption{Her ses Dijkstra algoritmen i praksis: De korteste ruter ud af hver træ beregnes indtil et mål rammes.}
		\label{dia:dijkstra1}
	\end{center}
\end{figure}

Denne algoritme er god til at finde den korteste vej til et mål, når den ikke ved hvilken vej den skal bevæge sig imod på forhånd.
