\rhschapter{Pathfinding algoritmer}

Følgende hovedspørgsmål er blevet stillet:
Hvilke pathfinding algoritmer anvendes ofte i spiludvikling og på hvilke måder er de forskellige?

Artificial intelligence og pathfinding bruges i rigtig mange spil. Når en enhed i et spil skal vælge at handle ud fra hvad spilleren gør er der tale om AI\footnote{Artificial Intelligence}. Denne intelligens skal ofte navigere eller foretage valg ud fra hvad spilleren gør. Pathfinding i sig selv er også brugbar uden kunstig intelligens, hvis man f.eks. skal navigere spilleren i terræn med musen rundt om forhindringer. I StarCraft, som er et RTS\footnote{Real Time Strategy} spil, skal man bygge bestemte bygninger for at kunne producere bestemte enheder. Pathfinding kan også bruges til at finde ud af hvilke bygninger der skal bygges i hvilken rækkefølge, for at AI'en kan bygge en bestemt enhed. Se \textcite[193]{buckland}, \textcite[198]{MillingtonFunge}. 

Der findes flere forskellige former for pathfinding, nogle er node baseret og kan anvendes til abstrakt pathfinding. Andre former er node og grid baseret og kan bruges med 2 eller 3 akser, som i et 2 eller 3 dimensionalt koordinatsystem. Se \textcite[193-203]{buckland}.

De følgende sektioner vil afdække de mest anvendte pathfinding algoritmer inden for computer spil.

%DFS
\section{DFS}
Den første pathfinding algoritme vi starter med er Depth First Search eller DFS forkortet. Denne algoritme bruges til at søge i grafer opbygget af noder og edges. Algoritmen søger ved at gå så dybt ned i grafen som muligt til at starte med, og når den så kommer til en blindgyde, går den tilbage til en node længere oppe og fortsætter derfra. Dette fortsætter indtil hele grafen er udforsket eller målet er nået. Se figur \ref{dia:dfsgrafik} for en demonstration af måden hvorpå DFS's fremgangsmåde virker. Se også \textcite[210]{buckland}. 

Da DFS søger hele vejen til enden i en retning og så backtracker, gør dette den dårlig til at søge i store grafer med mange edges og noder.

\pic{pictures/dfs.png}{DFS's fremgangsmåde}{dia:dfsgrafik}

Implementeringen af DFS algoritmen bruger en stack data struktur. Denne er af typen LIFO\footnote{Last In First Out}, dvs. at nye elementer i stack'en bliver indsat i toppen, og når man henter, henter man fra toppen af stack'en.

%BFS
\section{BFS}
Breadth First Search eller BFS forkortet er en algoritme der bruges til at søge i grafer. Denne søger ved at udforske alle edges som er forbundet til startnoden og derefter udforske alle edges forbundet til de noder den fandt fra de første edges. Det vil sige den tager et ekstra niveau af edges hver gang. Måden hvorpå BFS virker gør at den ikke er god til at søge i store grafer, f.eks. en level/map indelt i tiles, da den vil søge alle tiles igennem i en radius udfra start området, indtil den finder hvad du søgte efter. Se figur \ref{dia:bfsgrafik} for demonstration af hvordan BFS's fremgangsmåde fungerer.

\pic{pictures/bfs.png}{BFS's fremgangsmåde}{dia:bfsgrafik}

Måden hvorpå BFS søger i grafen, gør at hvis grafen har mange noder og edges, bliver BFS meget langsom. Se \textcite[230]{buckland}, \textcite{amitastar}.

Implementeringen af BFS algoritmen bruger en queue data struktur, som er af typen FIFO\footnote{First In First Out}. Det vil sige at nye elementer bliver tilføjet til bunden af queue'en, og når man ber om et element fra den, får man et fra toppen.

%Dijkstra
\section{Dijkstra}
Professor Edsger Wybe Dijkstra har bidraget med mange nyttige ting til computer videnskab. En af disse kaldes Dijkstra's Algorithm og er en pathfinding algoritme der gør brug af shortest path trees. Som man kan se på figur \ref{dia:shortest_path_trees}, har en edge imellem noder en kost associeret med at vælge den. Derved kan man lave et træ af veje fra node til node, og beregne hvilken vej der er den billigste(korteste) at bruge. Se \textcite[233]{buckland}, \textcite{amitastar}, \textcite[204]{MillingtonFunge}.

\begin{figure}
	\begin{center}
		\includegraphics[width=0.49\linewidth]{pictures/pathfinding/shortest_path_tree}
		\caption{Shortest Path Trees: Deler sig hver gang der er flere retninger. Hvis de rammer den samme node, er det den korteste af dem der fortsætter.}
		\label{dia:shortest_path_trees}
	\end{center}
\end{figure}

Hvis man kigger på implementeringen af Dijkstra's Algorithm i figur \ref{dia:dijkstra1}, kan man se at den laver en cirkelformet søgning ud fra startpunktet til den møder målet. Læg også mærke til at ingen af vejene krydser hinanden, for når algoritmen rammer en node, der allerede er i træet, bliver begge to evalueret, og kun den korteste af de to vil der blive brugt til videre beregninger. Læg også mærke til at det er dyrere at bevæge sig diagonalt end horisontalt og vertikalt med algoritmen.

\begin{figure}
	\begin{center}
		\includegraphics[width=0.49\linewidth]{pictures/pathfinding/dijkstra1}
		\includegraphics[width=0.49\linewidth]{pictures/pathfinding/dijkstra2}
		\caption{Her ses Dijkstra algoritmen i praksis: De korteste ruter ud af hver træ beregnes indtil et mål rammes.}
		\label{dia:dijkstra1}
	\end{center}
\end{figure}

Denne algoritme er god til at finde den korteste vej til et mål, når den ikke ved hvilken vej den skal bevæge sig imod på forhånd.


%AStar
\section{A*}
A*\footnote{A* udtales "ay-star"} er en forbedret Dijkstra søgning, hvor man i forvejen ved hvilken retning man skal bevæge sig imod. Denne algoritme er lidt mere kompleks og kræver f.eks. et koordinat system med lidt flere regler/guidelines at følge. Dette gør at vi kan estimere hvilken retning man skal bevæge sig i. Man lægger en ekstra kost på at bevæge sig væk fra sit mål. Er forsøgene fra startpunktet dyrere at foretage end den billigste fundne, opgives videre søgning ud af de træer. Mødes der en forhindring på vej mod målet, splittes træt igen, og der forsættes ud af begge træer indtil den ene igen bliver dyrere at bevæge sig ud af end den anden. Bliver den nuværende søgning dyrere end de gamle træer, der er opgivet, forsættes søgning ud af disse træer igen. Som det kan ses i figur \ref{dia:astar1}, spares der processorkraft ved ikke at udforske flere nodes ud af træer som bevæger sig længere væk fra målet. Se \textcite[241]{buckland}, \textcite{amitastar}, \textcite[215]{MillingtonFunge}.

\begin{figure}
	\begin{center}
		\includegraphics[width=0.49\linewidth]{pictures/pathfinding/astar1}
		\includegraphics[width=0.49\linewidth]{pictures/pathfinding/astar2}
		\caption{A* en modificeret Dijkstra hvor vi kender målets position, og derved kan estimere hvilken retning der er kortest at bevæge sig i mod. Derved kan man forhindre at bruge processorkraft på unødvendige veje.}
		\label{dia:astar1}
	\end{center}
\end{figure}

Hvis der nu er flere mål af samme type som man kan tage hen til, f.eks. at finde den nærmeste health powerup, kan man stadig bruge A*. Men det kræver at algoritmen har adgang til disse positioner. For at finde den nærmeste node kan man bruge en simpel beregning som kaldes Manhattan Distance Heuristic. Ved at sammenligne start positionens x og y, med alle mulige powerup positioner, kan man udregne en forskel i x og y afstand imellem de 2 positioner, se figur \ref{dia:manhattan}. Denne afstand målt i x og y lægges sammen, og så har man en kost fra start positionen til hver powerup. Med denne metode kan man vælge det mindste tal, og så har man den nærmeste node, dog uden hensyn til forhindringer. Man kan vælge at lave beregningerne på de nærmeste N antal, og vælge den som så returnere den korteste vej.

\begin{figure}
	\begin{center}
		\includegraphics[width=0.49\linewidth]{pictures/pathfinding/manhattan}
		\caption{Manhattan Heuristics: afstanden x + y imellem de to punkter anvendes som en kost. Derved den tætteste af flere punkter hurtig beregnes ud fra et start punkt.}
		\label{dia:manhattan}
	\end{center}
\end{figure}

%Avendelses områder
\section{Anvendelses områder}

Hvilke pathfinding algoritmer er bedst i hvilke situationer?
A* og Dijkstra egner sig begge til pathfinding i et level/map. Hvilken der er bedst at bruge kommer an på hvilken situation det er, f.eks. hvis du vil flytte en enhed i et RTS såsom StarCraft, kan A* bruges til at finde den korteste vej til destinationspunktet. A* søger i retning af destinationen se figur \ref{dia:astar1}. Hvis man skal finde den nærmeste af en type f.eks. nærmeste healthbox i et level/map er dijkstra bedst da denne laver en cirkelformet søgning ud fra startnoden se figur \ref{dia:dijkstra1}. 

A* kan kombineres med Manhattan Distance Heuristic til at finde den nærmeste healthpack. Dette har dog ulæmpen at hvis der er en væg imellem dem, f.eks. at den er i et andet rum, vil den stadig se den som den nærmeste healthpack. Man kan med Manhattan Distance Heuristic lave en liste over f.eks. de 5 nærmeste healthpacks, og så beregne en A* path til hver af dem, og så vælge den korteste fundet. Dette kan være hurtigere end Dijkstra, hvis der er langt ud til den nærmeste healthpack. Er man i den situation at AI'en ikke har adgang til healthpack'enes positioner, kan A* ikke bruges og Dijkstra vil være den hurtigste til at finde den nærmeste healthpack.

DFS kan bruges i spil som tilfældigt genererer baner, f.eks. Diablo II, for at teste om spilleren kan bevæge sig fra et start punkt til et slut punkt. Dette sikrer at den genererede bane kan bruges, og ikke har nogle ting som gør at spilleren sidder fast og ikke kan komme videre.

BFS er ikke særlig effektiv i forhold til Dijkstra, hvis der er kost associeret med valg, men den tager kortere tid at implementere, fordi at det er en mere simpel algoritme. Den er så også hurtigere end Dijkstra til situationer hvor der ikke er en kost associeret med valg.
BFS kan bruges i situationer, hvor man gerne vil finde ud af hvilke bygninger, der skal bygges først i f.eks. StarCraft, før du kan bygge en avanceret unit. Derved findes alle kravene med BFS.

%Afrunding
\section{Afrunding}
Vi har i denne artikel set på de algoritmer der anvendes ofte i spiludvikling, og hvordan de er forskellige fra hinanden.
Derefter har vi diskuteret i hvilke sammenhænge hver algoritme er brugbare, og givet et bud på hvor hver algoritme egner sig bedst.