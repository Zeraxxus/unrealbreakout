\section{Versionsstyring}
\textit{Af Bjarne Kristensen}\newline
Versionsstyring er et værktøj, der gør det nemt for flere personer at arbejde på de samme filer, og gør at der er backups man kan vende tilbage til. Ved versionsstyring er der altid mulighed for at tage en ældre version af filerne, hvis der skulle være
noget som er gået galt i en nyere version. Når der er to eller flere som sidder og arbejder på samme program, gør versionsstyring det muligt at sammenflette text baserede filer automatisk, men også manuelt skulle der være konfikter på linier.

Vi har anvendt versionstyring til udarbejdelsen af rapporten, men ikke spillet. Rapporten består af mange rå tekst filer, og disse er optimale at anvende versionstyring på. Spillet består til gengæld af blueprints og levels til Unreal Engine 4, og ved disse har vi ikke selv kontrol over tekst linier som ved ren kode. Derfor kan vi ikke bruge vores egne versionsprogrammer til at sammenflette koden mellem 2 udgaver af den samme level eller blueprint. Unreal engine har et versionstyring integreret som hedder Perforce, så hvis man skal bruge det skal man tilmelde sig en tjeneste der passer til det. Eller selv betale for at sætte en server op og betale licenser. Det virker ikke til at være en færdig feature i Unreal Engine 4 endnu, og det ser ud til at der ofte er spærgsmål på deres forum om hvordan det virker og at det crasher eller ikke gør noget\cite{blueprintmerge}.
Vi har valgt at bruge Github som depot for vores rapport tekst, da det er gratis at anvende ved offentlige projekter. Skal koden være privat skal man betale for en opgraderet bruger\cite{githubterms}. Github har også selv deres egen git klient som man kan bruge, men man kan også bruge andre programmer der understøtter git protokollen.