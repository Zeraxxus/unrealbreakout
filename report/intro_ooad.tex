\section{Objekt Orienteret Analyse og Design}
Efter problemformuleringen, afgrænsningen  og spil typen; breakout var bestemt, satte udviklingsteamet sig sammen og lavede et analysediagram. Analysediagrammet er et godt værktøj til, at få alle udviklere på samme spor og tankegang omkring projektet. Analysediagrammet lister produktets objekter, og deres relationer til hinanden. På denne måde fik udviklingsholdet snakket sammen om forholdet imellem de forskellige objekter, hvilket leder til en mere hensigtsmæssig udviling. Da hele udviklings holdet har samme tankegang omkring hvad hvert objekt indebærer. Efter dette blev use cases udviklet, som beskriver hvilke funktioner brugeren af produktet har adgang til. Use case-værktøjet udpensler derfor hvilke funktioner produktet skal indeholde, set fra brugerens synspunkt. Dette giver udviklergruppen en sans for hvilke funktionaliteter har en sammenhæng mellem objekter. For at gøre dette helt klart, udvikles en hændelses- og funktionstabel. Disse værktøjer udmærker sig ved, at klarificere Hvilke objekter i systemet der har fælles funktionalitet. Udover dette bruges værktøjerne også til at specificere funktionstyper, og kompleksiteten af disse. Til sidst i denne fase laves en systemdefinition, med værktøjet "FACTOR" til hjælp. Systemdefinitionen er en kort tekst der beskriver systemets facetter, dette kan bl.a. være systemets ansvar over for brugeren eller teknologien benyttet til at udvikle systemet. Efter brugen af alle disse værktøjer, er systemet til udviling klarlagt og udviklingen kan begyndes.