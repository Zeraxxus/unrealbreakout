\rhschapter{Forord}
I denne artikel gøres der brug af Harvard metoden for kildehenvisninger. Derved vil kilde henvisninger have følgende udseende: \textcite[1]{buckland}.
\newline\newline
Fodnoter vil også være at finde og vil have følgende udseende: f.eks.\footnote{forkortelse af ordene "for eksempel"}
\newline\newline
Der vil også være illustrationer i artiklen. Disse kaldes figurer og vil blive refereret i teksten som f.eks. figur 1.1.
\newline\newline
I denne artikel har vi brugt \textcite{buckland} som hovedkilde. Da vi skal diskutere hvad der ofte bruges i spilbranchen er det også vigtigt at bogen ikke kun er akademisk men også praktisk. Bogen er læst og peer reviewed af AI programmører fra diverse store firmaer i branchen. Dette gør bogen meget mere troværdig i forhold til påstande gjort i den. Se bilag \ref{cha:citater} for citater.
\newline\newline
Da bogen er af lidt ældre dato har vi også fundet en nyere bog som \textcite{MillingtonFunge} og kilder på nettet som viser at det stadigvæk er disse algoritmer spiludviklere nævner og diskuterer.